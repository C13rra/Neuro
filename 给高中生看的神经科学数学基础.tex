% Options for packages loaded elsewhere
\PassOptionsToPackage{unicode}{hyperref}
\PassOptionsToPackage{hyphens}{url}
%
\documentclass[
]{article}
\usepackage{amsmath,amssymb}
\usepackage{iftex}
\ifPDFTeX
  \usepackage[T1]{fontenc}
  \usepackage[utf8]{inputenc}
  \usepackage{textcomp} % provide euro and other symbols
\else % if luatex or xetex
  \usepackage{unicode-math} % this also loads fontspec
  \defaultfontfeatures{Scale=MatchLowercase}
  \defaultfontfeatures[\rmfamily]{Ligatures=TeX,Scale=1}
\fi
\usepackage{lmodern}
\ifPDFTeX\else
  % xetex/luatex font selection
\fi
% Use upquote if available, for straight quotes in verbatim environments
\IfFileExists{upquote.sty}{\usepackage{upquote}}{}
\IfFileExists{microtype.sty}{% use microtype if available
  \usepackage[]{microtype}
  \UseMicrotypeSet[protrusion]{basicmath} % disable protrusion for tt fonts
}{}
\makeatletter
\@ifundefined{KOMAClassName}{% if non-KOMA class
  \IfFileExists{parskip.sty}{%
    \usepackage{parskip}
  }{% else
    \setlength{\parindent}{0pt}
    \setlength{\parskip}{6pt plus 2pt minus 1pt}}
}{% if KOMA class
  \KOMAoptions{parskip=half}}
\makeatother
\usepackage{xcolor}
\setlength{\emergencystretch}{3em} % prevent overfull lines
\providecommand{\tightlist}{%
  \setlength{\itemsep}{0pt}\setlength{\parskip}{0pt}}
\setcounter{secnumdepth}{-\maxdimen} % remove section numbering
\ifLuaTeX
  \usepackage{selnolig}  % disable illegal ligatures
\fi
\usepackage{bookmark}
\IfFileExists{xurl.sty}{\usepackage{xurl}}{} % add URL line breaks if available
\urlstyle{same}
\hypersetup{
  hidelinks,
  pdfcreator={LaTeX via pandoc}}

\author{}
\date{}

\begin{document}

\section{0.高中生也能看懂的的神经科学数学基础}\label{0ux9ad8ux4e2dux751fux4e5fux80fdux770bux61c2ux7684ux7684ux795eux7ecfux79d1ux5b66ux6570ux5b66ux57faux7840}

\tableofcontents

\subsection{2024-07-14更新:}\label{2024-07-14ux66f4ux65b0}

\begin{itemize}
\item
  我本来以为两三个小时就能打完我这周学的东西了,结果我打了三个小时,却连生物物理学的部分都没打完,这才是我这周学的大概
  20\%。不仅如此,我觉得我写的东西貌似也没什么启发性\ldots\ldots 问题出在哪里?我不明白。我想公式的推导大概可以不写在文章里面,而是转用引用的方式写作。到底更应该注重什么东西?我不知道。

  \begin{itemize}
  \item
    我想首先的问题在于:我没有搞清楚这篇文章和教科书的区别------这篇文章应该和教科书相去甚远。文章的特点主要是记录了我的学习路径。系统学习这些课程的人可能会碰到一些问题,我的希望是他们看到这篇文章后能在短的时间内克服这些问题。
  \item
    受众是谁?主要是想要学习计算神经科学的高中生。
  \end{itemize}
\item
  知乎的文字编辑器实在是不敢恭维。导入 docx 吞公式,导入 md
  吞图片\ldots\ldots 图床真是烦人。
\item
  我把打了三个多小时的第一版删了,换了个诡异的的写作风格,图片也都去掉了,不知道合不合读者的胃口。
\end{itemize}

\subsection{0.1.为什么写这个东西?}\label{01ux4e3aux4ec0ux4e48ux5199ux8fd9ux4e2aux4e1cux897f}

笔者写这篇文章的时候刚刚高考结束完一个月,未来决定学习计算神经科学。我想要尽早参与神经科学的科研学习,但同时我的基础课也都没有学完。

我之前有个想法是一定要打好基础才能去进行下一步的学习,而这个所谓的
\textbf{打好基础}
到底到什么度为止,其实我并没有怎么想过,只是觉得学得越深越好越多越好,结果却做了很多的无用功。\textsubscript{这也是我的高中老师和竞赛教练一直有的想法。}

面对这个难题,Philip
Nelson的《生物物理学:能量、信息、生命》中的引言很启发我:

\begin{quote}
我认识到本科教育使我到大学最后一年(甚至更晚)才能接触到大量基本概念\ldots\ldots 尚未获得全貌,人们就开始小心翼翼地建造复杂的数学大厦\ldots\ldots 很多本科生在第一年就开始做研究,他们需要及早知道概况。
\end{quote}

PiKaChu345的\href{https://www.bilibili.com/video/BV1qb421n7qK/?spm_id_from=333.999.0.0&vd_source=8acc004888d1e147ebe6458188771c23}{视频}也在这个问题上启发了我:

\begin{quote}
认为要多读书的人往往认为\textbf{只有}到达前沿\textbf{才能}开始研究。他们潜意识里认为在到达前沿之前所有的问题都已经被人们解决了;只要到达前沿,自动学会研究。不管之前有没有受过做研究的教育。

认为要多做题的人往往认为要做大量的题目,打好坚持的基础。他们潜意识里认为\textbf{只要}自己的基础足够数量,\textbf{就}可以做出好的研究。

所以为什么不能像做研究一样学习知识呢?
\end{quote}

同时我也越来越意识到发现问题的重要性。之前的高中同学有几个总是能提出具有启发性的问题,这让我非常佩服。

我希望写这样一篇文章------能让没有基础的人们尽可能轻松地阅读,同时以研究感兴趣的问题作为文章的主线,而不是处处按照教材的顺序。

这篇文章是我边学边写的,势必会有很多问题和错误。但随着之后的学习,这些前面的问题和错误会被修正。也希望能吸引更多的人们来学习或指点。

我希望那个花费大约 100
个小时来写作这篇文章,在每周的周日更新写两个小时。今天是
2024-07-08,不出意外的话,明年的今天就可以完成这篇文章了。

\subsection{0.2.什么是计算神经科学?}\label{02ux4ec0ux4e48ux662fux8ba1ux7b97ux795eux7ecfux79d1ux5b66}

\textbf{Caution}\\

\begin{enumerate}
\def\labelenumi{\arabic{enumi}.}
\item
  计算神经科学在研究什么问题?
\item
  计算神经科学有哪些研究成果?
\end{enumerate}

\section{1.理解单个神经元:Hodgkin-Huxley
方程}\label{1ux7406ux89e3ux5355ux4e2aux795eux7ecfux5143hodgkin-huxley-ux65b9ux7a0b}

高中我们就已经学习了神经元的静息电位和动作电位:

\begin{itemize}
\item
  静息电位的时候 \(K^+\) 外流-\textgreater 外正内负
\item
  动作电位的时候 \(Na^+\) 内流-\textgreater 外负内正。
\end{itemize}

高中生的读者可能觉得这样就足够了,但其实这背后还有很多的问题没有解决:

\textbf{Caution}\\

\begin{enumerate}
\def\labelenumi{\arabic{enumi}.}
\item
  直观的看,\(V_m\) 的改变来自于 \(G_K\)、\(G_{Na}\)
  等\textbf{离子电导}的改变。不过电位到底为什么改变?离子电导由什么来决定?又因为什么而产生变化?我们能不能得知这些变化可能是什么形式的?
\item
  在\textbf{静息电位}的时候,细胞显然是处于一个\textbf{有别于外界的稳态}之中------\(V_m\)
  稳定在大约 \(-70\rm mV\) 左右而不是 \(0\)。这是由什么导致的呢?
\item
  \textbf{动作电位} 是如何产生的?又是如何 \textbf{传导} 的?我们都知道
  \textbf{\href{https://en.wikipedia.org/wiki/Myelin}{髓鞘}}
  的产生让有颌类脊椎动物的神经传导速度大大加快了,这是如何做到的?乌贼没有髓鞘,但它的轴突又粗又长,轴突的物理特征会怎么样影响动作电位的传导?
\end{enumerate}

\subsection{1.1离子电导}\label{11ux79bbux5b50ux7535ux5bfc}

\begin{itemize}
\item
  电位到底为什么改变?
\item
  离子电导由什么来决定?
\item
  离子电导因为什么而产生变化?我们能不能得知这些变化可能是什么形式的?
\end{itemize}

\subsection[1.2静息电位:电化学平衡]{\texorpdfstring{1.2静息电位:电化学平衡\footnote{《生物物理学:能量、信息、生命》\(\S4,\S7.4,\S11.1\)}\footnote{《费曼物理学讲义》
  \(\S 43\)}}{1.2静息电位:电化学平衡}}\label{12ux9759ux606fux7535ux4f4dux7535ux5316ux5b66ux5e73ux8861ux5cux255E2ux5cux255E3}

静息电位的产生源自于 \textbf{扩散} 和 \textbf{电场力作用下运动}
的平衡------或者说 \textbf{电化学平衡}。

这需要一些简单的物理知识,但毕竟这并不是生物物理学,所以我们并不需要细究背后的物理学机制。

\begin{enumerate}
\def\labelenumi{\arabic{enumi}.}
\item
  我们知道\textbf{扩散}的本质是无规行走。那么如何定量地描述无规行走?又如何定量地描述扩散?
\item
  我们知道粒子和溶液分子碰撞会导致\textbf{摩擦耗散}。这让溶液中粒子在外力下的运动不同于理想中的匀加速运动。那么碰撞的结果具体是什么?又如何定量地计算这种摩擦耗散?粒子在\textbf{电场力}
  和 溶液摩擦耗散作用下的运动如何定量地描述?
\item
  如何推导出\textbf{电化学平衡}?
\end{enumerate}

\subsubsection{1.2.1扩散:无规行走}\label{121ux6269ux6563ux65e0ux89c4ux884cux8d70}

\textbf{Important}\\

关于无规行走的结果,我们要推出来一个重要的结论是:

\[E[x(t)^2]=2Dt\]

关于公式的一个简单易懂的推导见
\emph{《生物物理学:能量、信息、生命》\(\S4.1.2\)}\footnote{《生物物理学:能量、信息、生命》\(\S4,\S7.4,\S11.1\)}
。推导这个结果并不需要任何超出高中的数学知识,只需要知道随机变量的运算法则就可以了。书中以下面的例子推导了位移
\(x\) 的二阶矩的均值 \(E[x_n^2]\) 与位移次数 \(n\) 的正比例关系:

\begin{quote}
假设这是一个\textbf{离散}的\textbf{一维}无规行走,假设一个粒子,从原点开始运动,每一次运动有一半的概率向
\(x\)-轴 正方向移动\(l\),有一半的概率向负方向移动 \(l\)。它在 \(n\)
次运动后的位移记为 \(x_n\) 。位移 \(L\) 也是一个随机变量:

\[P(L=l)=\frac{1}{2},P(L=-l)=\frac{1}{2}\]
\end{quote}

试着想象一下!书中的推导实际上仅仅用到了独立随机变量的期望乘积法则和

\[E[x_n]=0\]

就通过累加法得到了:

\[E[x_n^2]=nL^2\]

你可能会说这不过是二项分布而已,关于这个过程的全部细节都可以推出来。确实如此。不过我们得到的结论可不仅限于二项分布。实际上对于任意分布的
\(L\),通过简单的变换都可以得到:

\[E[(x_N-NE[L])^2]=NE[L^2]\]

这也是\emph{《生物物理学:能量、信息、生命》\(\S4.1.3\)}\footnote{《生物物理学:能量、信息、生命》\(\S4,\S7.4,\S11.1\)}
的结果。

\begin{center}\rule{0.5\linewidth}{0.5pt}\end{center}

\textbf{Important}\\

虽然这里只提到了

\[E[x(t)^2]=2Dt\]

不过很显然,对于 \(y\),\(z\) 也是一样的。对于三维的无规行走:

\[E[r^2]=E[x^2]+E[y^2]+E[z^2]=6Dt\]

\begin{center}\rule{0.5\linewidth}{0.5pt}\end{center}

关于无规行走还有一个有趣的例子有关高分子的尺寸和分子量,见\emph{《生物物理学:能量、信息、生命》\(\S4.3.1\)}\footnote{《生物物理学:能量、信息、生命》\(\S4,\S7.4,\S11.1\)}

在这个例子中我们得到:

\[E[r_N^2]=3NL^2\propto M\]

高分子的半径的平方和其分子量大致成正比。

\begin{center}\rule{0.5\linewidth}{0.5pt}\end{center}

\textbf{Note}\\

扩散在亚细胞层面非常迅速:

\begin{itemize}
\item
  对于室温下水中的小分子

  \[D\approx 10^{-9}{\rm m^2 s^{-1}}=1{\rm \mu m^2ms^{-1}}\]
\item
  在半径为 \(1{\rm \mu m}\)
  的细菌内,小分子从中心均匀扩散到整个细胞的用时大概只需要
  \(0.2 {\rm ms}\)。
\end{itemize}

\subsubsection{1.2.2.粘性摩擦系数:溶液的摩擦耗散和电场力下运动}\label{122ux7c98ux6027ux6469ux64e6ux7cfbux6570ux6eb6ux6db2ux7684ux6469ux64e6ux8017ux6563ux548cux7535ux573aux529bux4e0bux8fd0ux52a8}

\textbf{Caution}\\

摩擦是什么?

\begin{quote}
Gilbert:好的。你说过,由于热运动,处于室温的气体不会落到地板上。那么,它们为什么不会因为摩擦而减速并最终停止(然后落向地板)呢?\\
Sullivan:啊,那不可能,因为能量是守恒的。所有气体分子之间仅发生弹性碰撞,就像一年级物理课中学过的弹球那样。
\\
Gilbert:哦?那什么是摩擦呢?如果在比萨斜塔上丢下两个球,由于摩擦,轻的会落得慢一些。所有人都知道机械能并不守恒它最终转化为热能。\\
Sullivan:哦,嗯\ldots.\\
你看,一知半解对于两位虚拟的科学家来说真是件危险的事。假如不是扔一个小球,而是把一个气体分子以极大速度射进房间,比如100倍于平均速度(确实可以用粒子加速器做到这一点),会发生什么呢?

摘自《生物物理学:能量、信息、生命》\(\S3.2\)\footnote{《生物物理学:能量、信息、生命》\(\S4,\S7.4,\S11.1\)}
\end{quote}

我们先回顾一下高中物理学过的知识:

一个典型的例子是物体自由落体时受到的来自空气的平方阻力:

\[f=Cv^2\]

当物体以稳定的速度下落的时候:

\[Cv_{ss}^2=mg\]

解得的稳定速度:

\[v_{ss}=\sqrt\frac{mg}{C}\]

引入这个例子的原因并不是想说明具体的数学细节,而是想说明最终的稳定速度是由外力
\(F\) 和摩擦共同决定的。在这个例子中 \(f\) 是 \(v\)
的函数。不过也并不一定如此。

\textbf{Important}\\

和上一个例子不同,这里甚至都没有阻力 \(f\) 出现
。\textbf{粘性摩擦耗散}源于物理实体和周围热运动的流体分子的随机碰撞。随机碰撞的摩擦耗散的结果是粒子在每次碰撞之后的初速度满足

\[E[v_0]=0\]

所以在外力驱动下,一维的匀加速运动的初速度为 \(0\),

\[E[x]=\frac{f}{2m}(\Delta t)^2\]

粒子的平均速度为

\[v_{diff}=\frac{E[x]}{\Delta t}=\frac{F\Delta t}{2m}\]

\textbf{Important}\\

而其中 \(m\),\(\Delta t\) 都是粒子本身的特性。而和 \(F\)
无关。因此可以设 \(\zeta\) 为\textbf{粘性摩擦系数}:

\[v_{diff}=\frac{F}{\zeta}\]

在这个模型中:

\[\zeta = \frac{2m}{\Delta t}\]

\textbf{Note}\\

\textbf{Stokes 定律}给出了在低雷诺数流体中,球形物体的粘性摩擦系数
\(\zeta\):

\[\zeta=6\pi\eta R\]

\textbf{Caution}\\

值得一提的是这里的

\[\zeta=\frac{2m}{\Delta t}\]

并不一定,会随着模型的改变而改变。不过我们接下来要推出来的
\textbf{Einstein 关系} 却是一定的。

\textbf{Caution}\\

另外,引入一个概率分布的 \(\Delta t\) 并不会改变

\[v_{diff}=\frac{F}{\zeta}\]

\textbf{Einstein 关系}也不会改变:

\[\zeta D=k_BT\]

不过在这里,\(\zeta\) 的定义有所不同:

\[\zeta = \frac{m}{\tau}\]

其中 \(\tau\) 是平均碰撞时间。

\subsubsection{1.2.3.统计物理学的基本知识}\label{123ux7edfux8ba1ux7269ux7406ux5b66ux7684ux57faux672cux77e5ux8bc6}

根据1.2.1

\[D=\frac{L^2}{2\Delta t}\]

根据1.2.2

\[\zeta = \frac{2m}{\Delta t}\]

根据1.2.3

\[(\frac{L}{\Delta t})^2=E[v^2]=\frac{k_BT}{m}\]

我们就推导出了

\textbf{Important}\\

\textbf{Einstein 关系}:

\[\zeta D=k_B T\]

\subsubsection{1.2.4.Einstein
关系的推导}\label{124einstein-ux5173ux7cfbux7684ux63a8ux5bfc}

\subsection{1.3.描述动作电位的产生和传播}\label{13ux63cfux8ff0ux52a8ux4f5cux7535ux4f4dux7684ux4ea7ux751fux548cux4f20ux64ad}

\end{document}
